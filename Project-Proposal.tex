\documentclass[11pt,]{article}
\usepackage{lmodern}
\usepackage{amssymb,amsmath}
\usepackage{ifxetex,ifluatex}
\usepackage{fixltx2e} % provides \textsubscript
\ifnum 0\ifxetex 1\fi\ifluatex 1\fi=0 % if pdftex
  \usepackage[T1]{fontenc}
  \usepackage[utf8]{inputenc}
\else % if luatex or xelatex
  \ifxetex
    \usepackage{mathspec}
  \else
    \usepackage{fontspec}
  \fi
  \defaultfontfeatures{Ligatures=TeX,Scale=MatchLowercase}
    \setmainfont[]{Arial}
\fi
% use upquote if available, for straight quotes in verbatim environments
\IfFileExists{upquote.sty}{\usepackage{upquote}}{}
% use microtype if available
\IfFileExists{microtype.sty}{%
\usepackage{microtype}
\UseMicrotypeSet[protrusion]{basicmath} % disable protrusion for tt fonts
}{}
\usepackage[margin=.5in]{geometry}
\usepackage{hyperref}
\hypersetup{unicode=true,
            pdfborder={0 0 0},
            breaklinks=true}
\urlstyle{same}  % don't use monospace font for urls
\usepackage{natbib}
\bibliographystyle{plainnat}
\IfFileExists{parskip.sty}{%
\usepackage{parskip}
}{% else
\setlength{\parindent}{0pt}
\setlength{\parskip}{6pt plus 2pt minus 1pt}
}
\setlength{\emergencystretch}{3em}  % prevent overfull lines
\providecommand{\tightlist}{%
  \setlength{\itemsep}{0pt}\setlength{\parskip}{0pt}}
\setcounter{secnumdepth}{5}

%%% Use protect on footnotes to avoid problems with footnotes in titles
\let\rmarkdownfootnote\footnote%
\def\footnote{\protect\rmarkdownfootnote}


  \title{}
    \author{}
    \date{}
  
\usepackage{booktabs}
\usepackage{longtable}
\usepackage{array}
\usepackage{multirow}
\usepackage{wrapfig}
\usepackage{float}
\usepackage{colortbl}
\usepackage{pdflscape}
\usepackage{tabu}
\usepackage{threeparttable}
\usepackage{threeparttablex}
\usepackage[normalem]{ulem}
\usepackage{makecell}
\usepackage{xcolor}

%%%%%%%%%%
% personal preamble edits here
%%%%%%%%%%
\pagenumbering{gobble}

% can toggle this for Helvetica
%\usepackage{helvet}
%\renewcommand{\familydefault}{\sfdefault}

% \titlespacing*{\paragraph}{0pt}{2pt}{1em}

% set section numbering/lettering
% tips for seccntformat: https://tex.stackexchange.com/questions/95896/how-to-format-subsection-title-without-packages
\makeatletter
\def\@seccntformat#1{%
  \expandafter\ifx\csname c@#1\endcsname\c@section\else
  \expandafter\ifx\csname c@#1\endcsname\c@paragraph\else
  \csname the#1\endcsname\quad
  \fi\fi}
  
  % stucture for these commands: https://texfaq.org/FAQ-atsigns
  \renewcommand\section{
  \@startsection{section}{1}{\z@}
    {-3.5ex \@plus -1ex \@minus -.2ex}
    {1.0ex \@plus.2ex} %reduce space below section (was 1.5ex)
    {\normalfont\normalsize\bf\uppercase}} %modify font style
    
  \renewcommand\subsection{
  \@startsection{subsection}{2}{\z@}
    {-1.5ex\@plus -1ex \@minus -.2ex}%reduce space above subsection (was -3.25ex)
    {0.5ex \@plus .2ex}%reduce space below subsection (was 1.5ex)
    {\normalfont\normalsize\bf}} %modify font style
    
  \renewcommand\subsubsection{
  \@startsection{subsubsection}{3}{\z@}
    {-1.0ex\@plus -1ex \@minus -.2ex}%reduce space above subsubsection (was -3.25ex)
    {0.5ex \@plus .2ex}%reduce space below subsubsection (was 1.5ex)
    {\normalfont\normalsize\bf}} %modify font style
    
  \renewcommand\paragraph{
  \@startsection{paragraph}{4}{\z@}
    {-0.5ex\@plus -1ex \@minus -.2ex}%reduce space above paragraph (was -3.25ex)
    {-1.5ex \@plus .2ex}%convert space below paragraph to an indent (was 1.5ex)
    {\normalfont\normalsize\bf}} %modify font style    
\makeatother

\renewcommand\thesubsection{\Alph{subsection}.}
\renewcommand\thesubsubsection{\thesubsection\arabic{subsubsection}.}

% reduce spacing at the top of lists
\usepackage{enumitem}
\setlist{topsep = 2pt}

% allow text to wrap around figures
\usepackage{graphicx}
\usepackage{wrapfig}

%%%%%%%%%%

\begin{document}

\hypertarget{project-summaryabstract}{%
\section{Project Summary/Abstract}\label{project-summaryabstract}}

This project aims to leverage the comprehensive research database
established by the Center for International Blood and Marrow Transplant
Research® (CIBMTR) for hematopoietic cell transplantation (HCT). The
project is divided into three specific aims: a descriptive analysis of
enrolled patients to understand demographic and clinical
characteristics, a survival analysis focusing on the time from HCT to
seven different endpoints, and the development of an R Shiny application
for dynamic and interactive visualization of study results. This
initiative seeks to enhance understanding of HCT outcomes, identify
factors influencing survival post-transplant, and facilitate data
accessibility for clinicians and researchers alike, ultimately
contributing to improved patient care and outcomes in the field of
cellular therapies.

\pagebreak

\hypertarget{project-narrative}{%
\section{Project Narrative}\label{project-narrative}}

Sickle cell disease (SCD) is a group of inherited red blood cell
disorders affecting millions of people throughout the world. In someone
with Sickle Cell Disease (SCD), their red blood cells contain abnormal
hemoglobin, causing them to become misshapen, resembling a sickle. These
cells die prematurely, constantly decreasing the body's red blood cell
count. They can also block blood flow in small vessels, triggering pain,
and severe complications like infections, acute chest syndrome, and
stroke. Despite a variety of strategies including supportive care, drug
therapies, and red blood cell transfusions are able to potentially
alleviate symptoms and extend lifespan of SGD patients, allogeneic
hematopoietic cell transplantation (HCT) is the only established
potential cure for SCD. Nevertheless, post-HCT SCD patients continue to
face severe health challenges because HCT is associated with
life-threatening complications most of which occur within the first 2
years after transplantation such as graft-versus-host disease (GVHD) and
mortality. However, there is limited study on exploring features of
post-HCT SGD patients that affect the outcomes for time to significant
endpoints including last contact or death, graft failure, neutrophil
engraftment, platelet recovery, acute and chronic graft-vs-host disease
(GVHD), post-transplant lymphoproliferative disorder (PTLD), and second
malignancy. Moreover, no standardized data analysis platforms have been
established for clinical investigators to efficiently interact with and
generate insight from the database of post-HCT SCD patients.

\hypertarget{specfic-aims}{%
\section{Specfic Aims}\label{specfic-aims}}

\hypertarget{exploratory-data-analysis}{%
\subsection{Exploratory Data Analysis}\label{exploratory-data-analysis}}

We will perform exploratory analyses on variables included in the
datasets to obtain an overview of the baseline characteristics of
patients who received HCT and visually explore the features that are
potentially associated with the survival outcomes of the patients.
Specifically, the dataset categorizes the variables into 3 categories:
disease-related, patient-related, and transplant-related. By summarizing
information on variables in the dataset, we expect to learn the
distribution of baseline variables that could facilitate further
processing of the data, and provide initial insights on identifying
important predictors contributing to predicting the clinical outcome e
of patients.

\hypertarget{survival-analysis}{%
\subsection{Survival Analysis}\label{survival-analysis}}

We will perform a survival analysis to assess the time from HCT to 8
endpoints, including last contact or death, graft failure, neutrophil
engraftment, platelet recovery, acute and chronic graft-vs-host disease
(GVHD), post-transplant lymphoproliferative disorder (PTLD), and second
malignancy. These analyses will help identify key predictors of survival
outcomes which serve as potential factors that could enhance the quality
of care for patients undergoing HCT treatment.

\hypertarget{shiny-app-development}{%
\subsection{Shiny App Development}\label{shiny-app-development}}

Finally, an R Shiny application will be developed to not only make
findings interactive and accessible for clinical investigators but also
serve as a standardized platform to generate results accordingly with a
more updated database in the future. This interactive tool will allow
users to explore the data dynamically, visualize survival curves,
compare outcomes across different patient groups, and potentially
identify new areas for research or intervention. In order to enhance the
understanding and management of Sickle Cell Disease (SCD)
post-hematopoietic cell transplantation (HCT), we propose the
development of an R Shiny application, a crucial tool designed to
visualize and interpret complex datasets.

\hypertarget{research-strategy}{%
\section{Research Strategy}\label{research-strategy}}

\hypertarget{significance}{%
\subsection{Significance}\label{significance}}

Recognizing the significance of HCT regarding the curation of SGD and
the post-HCT potential health obstacles, our team aims to explore
crucial patient-related, disease-related, and transplant-related factors
contributing to clinical outcomes after HCT among SGD patients by
utilizing the database from Center for International Blood and Marrow
Transplant Research (CIBMTR) in 2021. The project initiative is to
specify the granularity for both evaluating success and subsequent
health threats post-HCT and assist clinical professionals with the
decision on establishing a more comprehensive treatment plan for SGD
patients by taking key patient-related, disease-related, and
transplant-related features into consideration.

\hypertarget{innovation}{%
\subsection{Innovation}\label{innovation}}

\begin{itemize}
\item
  Analysis Innovation

  For the survival analysis of this project, the targeted time-to-events
  include not only outcomes with implications of HCT failure such as
  ``time to GVHD'' and ``time to PTLD(Post-transplant
  lymphoproliferative diseases)'' but also implications of HCT success
  such as ``time to neutrophil engraftment'' and ``time to platelet
  recovery''. In this way, we are able to have a more comprehensive
  understanding to assess the HCT performance for SGD patients
\item
  Technology Innovation

  In this project, we plan to create a standardized and interactive
  visualization platform to illustrate the analysis results via an R
  shiny web application. Besides the ease of exploring and engaging with
  research outcomes, we also recognize the advantage of an R shiny web
  application to serve as a standardized tool for assisting future
  researchers to expand upon this project by feeding more recent data to
  this platform and obtaining more updated results.
\end{itemize}

\hypertarget{data-pre-processing}{%
\subsection{Data Pre-Processing}\label{data-pre-processing}}

After comparing the dataset from 2020 to 2021, our team has decided to
focus on the dataset from 2021 due to its enhanced data completeness and
inclusiveness of the dataset from 2020. To ensure the quality of
analysis, columns for variables excluding outcomes with over 20\%
missing data were removed. In addition, variables with unchanging values
were disregarded. In the end, we have the following numbers of variables
in addition to patient ID (see Appendix):

\begin{itemize}
\item
  11 patient-related variables
\item
  1 disease-related variable
\item
  10 transplant-related variables
\item
  8 outcome variables/endpoints
\end{itemize}

\hypertarget{specific-aim-1-descriptive-analysis-of-enrolled-patients}{%
\subsection{Specific Aim 1: Descriptive Analysis of Enrolled
Patients}\label{specific-aim-1-descriptive-analysis-of-enrolled-patients}}

\textbf{Hypothesis:} The exploratory data analysis aims to characterize
the distribution of baseline variables and to initially assess
associations between the distribution of baseline characteristics and
the occurrence of clinical endpoint events. Specifically, for each
categorical variable, we will calculate the count and proportion of each
value level, and for continuous variables, we will calculate the mean
and standard deviation. We will then examine the association between
occurrence of clinical endpoints (all-purpose death, signs of recovery)
and baseline variables using Logistic regression, and examine the
associations between explanatory variables by either Pearson's
Chi-squared test (categorical VS categorical), linear regression
(numerical VS numerical), and t-test or Wilcoxen test (numerical VS
categorical).

\textbf{Rationale:} Conducting exploratory analyses allows us to
understand the structure and the pattern in the data, and thus
facilitates the process of identification of potentially unbalanced
variables and the decision whether to keep these variables in downstream
analyses. By testing associations between outcome variables and response
variables, we could gain initial insights into characteristics that
might be clinically relevant in HCT patients.

\textbf{Experimental Approach:} Summary tables will be created by
processing the data with R package \texttt{dplyr} and presented with the
package \texttt{gt}. When examining associations between explanatory
variables, we will implement Pearson's Chi-squared test by the R
function \texttt{chisq.test}, linear regression by R function
\texttt{lm} , t test by \texttt{t.test} and Wilcoxen's test by
\texttt{wilcox.test}. Highly associated explanatory variables will be
further examined and the decision on exclusion of one of the highly
associated variables will be made. To fit Logistic regression to test
the association between explanatory variables and outcome variables, the
R function \texttt{glm} will be used.

\textbf{Interpretation of Results:} From the model fit, we could extract
predictors that are associated with significant p-values with a
pre-specified significance level (0.05), and consider them as important
predictors for the occurrence of specific clinical outcome.

\textbf{Potential Problems and Alternative Approaches: } The results of
the Logistic regression could be biased by unbalanced distribution of
predictors, and this could be mitigated by combining rare categories.

\hypertarget{specific-aim-2-survival-analysis}{%
\subsection{Specific Aim 2: Survival
Analysis}\label{specific-aim-2-survival-analysis}}

\textbf{Hypothesis:} The time from hematopoietic cell transplantation
(HCT) to various clinical endpoints (last contact or death, graft
failure, neutrophil engraftment, platelet recovery, acute and chronic
graft-vs-host disease (GVHD), post-transplant lymphoproliferative
disorder (PTLD), and second malignancy) significantly differ among
patient groups defined by specific demographic and clinical
characteristics.

\textbf{Rationale:} Understanding the time to key post-HCT outcomes is
crucial for several reasons.

\begin{itemize}
\item
  Firstly, it enables the prediction of patient prognosis by providing
  estimates on when significant post-transplant events might occur, thus
  informing both patients and clinicians about possible recovery paths
  and any potential complications that could arise. This knowledge is
  invaluable in setting realistic expectations and preparing for any
  necessary interventions.
\item
  Secondly, insights gained from analyzing the timing of events such as
  graft-versus-host disease (GVHD) are instrumental in guiding clinical
  decisions. They allow healthcare providers to customize
  post-transplant care according to the specific needs of each patient,
  potentially mitigating risks and enhancing recovery.
\item
  Lastly, by identifying the factors that influence the speed of
  recovery or the occurrence of delayed complications, adjustments can
  be made to the transplant process itself. This could involve selecting
  different donors, modifying conditioning regimens, or altering
  supportive care practices, all aimed at improving the overall success
  and safety of the transplant procedure.
\end{itemize}

\textbf{Experimental Approach:}

\begin{itemize}
\tightlist
\item
  Cox Proportional Hazards (CoxPH) Model: To adjust for potential
  confounders and assess the impact of various predictors on the time to
  each endpoint, CoxPH models will be fitted. A test of the proportional
  hazards assumption can be based on testing whether there is a linear
  relationship between the Schoenfeld residuals and some function of
  time. Variables will include patient demographics, disease
  characteristics, transplant-related factors, and pre-transplant
  comorbidities. We can conduct variable selection by analyzing the
  log-likelihood and corresponding p-values calculated using the
  \texttt{anova} function in the survival package for the specific model
  fit.\\
\item
  Kaplan-Meier (KM) Survival Curves: For each of the eight endpoints, KM
  survival curves will be generated to visualize the unadjusted
  probability of reaching each endpoint over time post-HCT.
\item
  *During this approach, the proportional hazard assumption will be
  verified. If the assumption fails to hold, random survival forest will
  be conducted instead.
\end{itemize}

\textbf{Interpretation of Results: } In our analysis using the Cox
Proportional Hazards model, we anticipate identifying several key
variables that significantly influence the various endpoints of interest
post-HCT, such as time to graft failure, neutrophil engraftment,
platelet recovery, and the onset of acute and chronic graft-vs-host
disease, among others. These variables could range from patient-related
factors (e.g., age, gender, underlying disease type, and stage), to
treatment-related factors (e.g., type of donor, conditioning regimen
intensity), to post-transplant care aspects (e.g., immunosuppression
protocol, infection prophylaxis).\\
Once these influential variables are identified, we plan to conduct a
stratification analysis to categorize patients into groups based on
their characteristics. For example, patients might be grouped by age
range, disease stage at transplant, or by the type of donor (related
vs.~unrelated) to see how these factors affect outcomes differently.\\
Following the stratification, we will plot Kaplan-Meier (KM) Survival
Curves for each group. These plots will visually represent the survival
probabilities over time for each stratified patient group, allowing us
to observe any significant differences in survival rates among the
groups. For instance, we may see distinct survival curves for younger
vs.~older patients or for those receiving transplants from matched
unrelated donors vs.~sibling donors.\\
The expectation is that these stratified KM Survival Curves will reveal
significant disparities in survival times across different patient
groups, highlighting the impact of the identified variables on patient
outcomes.

\textbf{Potential Problems and Alternative Approaches:} In tackling the
survival analysis for our study, we may encounter several challenges
that necessitate alternative approaches. One such challenge is the
violation of the proportional hazards assumption, a cornerstone of many
survival analysis methods. Should this occur, we would pivot to
employing time-varying covariate models or consider other survival
analysis techniques, such as accelerated failure time models, which do
not rely on this assumption. Another potential issue could arise from
having sparse data for certain outcomes, which might reduce the
statistical power of our analyses. In these instances, we might need to
aggregate similar categories or focus our analysis on the most
clinically relevant outcomes to ensure meaningful results. Furthermore,
the identification of complex interactions among predictors could
complicate the model's interpretability and accuracy. To address this,
machine learning methods like random forests or survival trees could be
employed, offering a more nuanced capture of these intricate
relationships. These approaches not only help us navigate potential
pitfalls but also enhance the robustness and relevance of our findings,
ensuring they contribute effectively to the field.

\hypertarget{specific-aim-3-development-of-an-r-shiny-application-for-results-visualization}{%
\subsection{Specific Aim 3: Development of an R Shiny Application for
Results
Visualization}\label{specific-aim-3-development-of-an-r-shiny-application-for-results-visualization}}

This application will serve as an interactive platform for clinicians,
researchers, and policy-makers, enabling them to dynamically explore and
analyze the wealth of patient-related, disease-related, and
transplant-related variables and their impact on post-HCT outcomes.

\textbf{Rationale:} The rationale behind this application is twofold.
Firstly, given the complexity and multi-dimensionality of the data,
traditional static methods of data presentation are insufficient for
capturing the nuanced relationships between variables such as age,
ethnicity, disease genotype, transplant type, and conditioning regimen.
An interactive tool will allow for a more comprehensive and tailored
exploration of these variables, fostering a deeper understanding of
their interplay and impact on patient outcomes. Secondly, this
application aims to democratize data access, allowing users to generate
custom analyses and visualizations that can inform clinical
decision-making and policy development.

\textbf{Experimental Approach: }The app will be implemented using R
Shiny, leveraging its capabilities for creating interactive, web-based
data visualizations. Key features will include the ability to filter and
stratify data based on specific variables like age group, disease
genotype, or transplant type, and visualize these in the form of
survival curves, bar charts, and heatmaps. For instance, users could
compare survival outcomes between different age groups or analyze the
impact of donor-recipient HLA matching on post-transplant complications.
This approach will enable users to interact with the data, providing
immediate visual feedback and insights.

\textbf{Interpretation of Results:} The results presented through the
Shiny app will offer valuable insights into the factors that influence
post-HCT outcomes in SCD patients. By enabling the exploration of
complex datasets in an intuitive manner, the app will not only aid in
the interpretation of current research findings but also spark new
hypotheses and areas for further study. It will facilitate a more
nuanced understanding of the SCD patient journey post-HCT, contributing
significantly to the field of hematology and transplant medicine.

\pagebreak

\hypertarget{appendix}{%
\subsubsection{Appendix}\label{appendix}}

\begin{longtable}[t]{|>{}ll>{}l|}
\caption{\label{tab:unnamed-chunk-6}List of Predictors}\\
\toprule
Category & Variable.name & Description\\
\midrule
Disease-related & subdis1f & Disease genotype\\
Patient-related & Dummyid & Unique patient identifier\\
Patient-related & flag\_lancet & Cases from 2019 Lancet Heamatology publication\\
Patient-related & flag\_blood & Cases from 2016 Blood publication\\
Patient-related & flag\_0601 & Cases from BMT CTN 0601\\
\addlinespace
Patient-related & age & Patient age at transplant, years\\
Patient-related & agegpff & Patient age at transplant, years\\
Patient-related & sex & Sex\\
Patient-related & ethnicit & Ethnicity\\
Patient-related & kps & Karnofsky/Lansky score at HCT\\
\addlinespace
Patient-related & hctcigpf & HCT-comorbidity index\\
Tranplant-related & donorf & Donor type\\
Tranplant-related & graftype & Graft type\\
Tranplant-related & condgrpf & Conditioning intensity\\
Tranplant-related & condgrp\_final & Conditioning regimen\\
\addlinespace
Tranplant-related & atgf & ATG/Alemtuzumab given as conditioning regimen/GVHD prophylaxis\\
Tranplant-related & gvhd\_final & GVHD prophylaxis\\
Tranplant-related & hla\_final & Donor-recipient HLA matching\\
Tranplant-related & rcmvpr & Recipient CMV serostatus\\
Tranplant-related & yeargpf & Year of transplant\\
\addlinespace
Tranplant-related & yeartx & Year of transplant\\
\bottomrule
\end{longtable}

\begin{longtable}[t]{|>{}ll>{}l|}
\caption{\label{tab:unnamed-chunk-6}List of Time-to-Event Variables}\\
\toprule
Category & Variable.name & Description\\
\midrule
Outcomes & intxsurv & Time from HCT to date of last contact or death, months\\
Outcomes & intxgf & Time from HCT to graft failure, months\\
Outcomes & intxanc & Time from HCT to neutrophil engraftment, months\\
Outcomes & intxplatelet & Time from HCT to platelet recovery, months\\
Outcomes & intxagvhd & Time from HCT to acute graft-vs-host disease, months\\
\addlinespace
Outcomes & intxcgvhd & Time from HCT to chronic graft-vs-host disease, months\\
Outcomes & intxptld & Time from HCT to PTLD, months\\
Outcomes & intxscdmal & Time from HCT to second malignancy, months\\
\bottomrule
\end{longtable}

\pagebreak

\begin{longtable}[t]{|>{}ll>{}l|}
\caption{\label{tab:unnamed-chunk-7}List of Outcome Varaibles (Exclude Time-to-Event)}\\
\toprule
Category & Variable.name & Description\\
\midrule
Outcomes & dead & Survival status at last contact\\
Outcomes & efs & Event-free survival (Graft failure or death are the events)\\
Outcomes & gf & Graft failure\\
Outcomes & dwogf & Death without graft failure\\
Outcomes & anc & Neutrophil engraftment\\
\addlinespace
Outcomes & dwoanc & Death without neutrophil engraftment\\
Outcomes & platelet & Platelet recovery\\
Outcomes & dwoplatelet & Death without platelet recovery\\
Outcomes & agvhd & Acute graft versus host disease, grades II-IV\\
Outcomes & dwoagvhd & Death without acute graft versus host disease, grades II-IV\\
\addlinespace
Outcomes & cgvhd & Chronic graft-vs-host disease\\
Outcomes & dwocgvhd & Death without chronic graft-vs-host disease\\
Outcomes & ptld & Post-transplant lymphoproliferative disorder (PTLD)\\
Outcomes & scdmal\_final & Secondary malignancy\\
\bottomrule
\end{longtable}

\clearpage
\nocite{*}

\bibliography{references.bib}


\end{document}
